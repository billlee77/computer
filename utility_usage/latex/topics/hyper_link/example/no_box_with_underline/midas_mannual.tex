\documentclass[12pt, letterpaper]{article}

\usepackage{times}
\usepackage{amsmath}
\usepackage{setspace}
\usepackage{listings}
\usepackage{fullpage}

\usepackage{hyperref}
\hypersetup{pdfborder= {0 0 0 0 0}, colorlinks =true, linkcolor=red, urlcolor=blue}




\lstset{basicstyle=\normalsize}

\special{papersize=8.5in,11in}

\newcommand{\longindent}{\hangindent=1.01cm}


\setlength{\parindent}{0cm}


% \hypersetup{
%     bookmarks=false,         % show bookmarks bar?
%     unicode=false,          % non-Latin characters in Acrobat’s bookmarks
%     pdftoolbar=true,        % show Acrobat’s toolbar?
%     pdfmenubar=true,        % show Acrobat’s menu?
%     pdffitwindow=false,     % window fit to page when opened
%     pdfstartview={FitH},    % fits the width of the page to the window
%     pdftitle={My title},    % title
%     pdfauthor={Author},     % author
%     pdfsubject={Subject},   % subject of the document
%     pdfcreator={Creator},   % creator of the document
%     pdfproducer={Producer}, % producer of the document
%     pdfkeywords={keyword1} {key2} {key3}, % list of keywords
%     pdfnewwindow=true,      % links in new window
%     colorlinks=false,       % false: boxed links; true: colored links
%     linkcolor=red,          % color of internal links
%     citecolor=red,        % color of links to bibliography
%     filecolor=magenta,      % color of file links
%     urlcolor=cyan           % color of external links
% }
% 







\title{Midas Installation Notes}

\author{Department of Physics \\ University of Regina}


\begin{document}
\maketitle

\doublespace

\begin{abstract}
This documentation is written to explain the installation procedure of the MIDAS data acquisition system. 
\end{abstract}




Note:
The basic steps of midas installation is explained under:

{\centering
\href{https://www.triumf.info/wiki/DAQwiki/index.php/Setup\_MIDAS\_experiment}{\underline{TRIUMF MIDAS Webpage}} \footnote{https://www.triumf.info/wiki/DAQwiki/index.php/Setup\_MIDAS\_experiment}\\

This documentation will only summarize the procedure. 
}

\section{Setup Before Installation}

%{\bf \noindent Step 1: Setup environment variable}

Edit .cshrc file and add the following

{\tt 
\begin{lstlisting}
    setenv LANG C
    setenv CVS_RSH ssh
    setenv MIDASSYS $HOME/packages/midas
    setenv ROOTSYS  $HOME/packages/root
    setenv MIDAS_EXPTAB $HOME/experiment/exptab
    # setup the MIDAS mserver
    switch (`hostname`)
    case larch*:
        unsetenv MIDAS_SERVER_HOST
        breaksw
    default:
        setenv MIDAS_SERVER_HOST larch.phys.uregina.ca:7071
    endsw
    # select 64-bit or 32-bit MIDAS and ROOT
    switch (`uname -i`)
    case i386:
       setenv PATH .:$MIDASSYS/linux-m32/bin:$PATH
       breaksw
    default:
       setenv ROOTSYS $HOME/packages/root
       setenv PATH .:$MIDASSYS/linux/bin:$PATH
    endsw
    setenv PATH .:$HOME/online/bin:$HOME/packages/roody/bin:$PATH
    setenv PATH .:$HOME/packages/roody/bin
\end{lstlisting}
}

% @ option will set according to the root directory installation, PLEASE DO NOT set \$ROOTSYS this way if the root is installed from the distribution, i.e RPMs.
% 
% * options are optional depending on user preferences. 

Note that {\tt MIDAS\_SERVER\_HOST} needs to be set accordingly. \texttt{ROOTSYS} should be set according to the actual ROOT directory on the computer, if ROOT is installed from Linux distribution, the do not set \texttt{ROOTSYS}.



\section{Packages checkout and compilation} 

Note: Root installation will not be covered here

\subsection{Install MIDAS}

{\tt 
\begin{lstlisting}
    $ cd $HOME/packages
    $ svn co svn+ssh://svn@savannah.psi.ch/repos/meg/midas/trunk midas 
      (Password "svn"). (password has to be entered twice)
    $ svn co svn+ssh://svn@savannah.psi.ch/repos/meg/mxml/trunk mxml
    $ cd midas
    $ make
    $ check that linux/bin/odbedit has been created
\end{lstlisting}
}


\subsection{Install ROOTANA}
 
{\tt 
\begin{lstlisting}    
    $ cd $HOME/packages
    $ svn checkout https://ladd00.triumf.ca/svn/rootana/trunk rootana 
      (say "p" to accept the ladd00 ssl certificate, 
	   use username "svn", password "svn")
    $ cd rootana
    $ make
\end{lstlisting}    
}


\subsection{Install ROODY}

{\tt 
\begin{lstlisting}
    $ cd $HOME/packages
    $ svn checkout https://ladd00.triumf.ca/svn/roody/trunk roody
    $ cd roody
    $ make
    $ $HOME/packages/roody/bin/roody
\end{lstlisting}    
}


\subsection{ROOT Configuration Comments}

Depending on how ROOT packages are installed into the computer, there are some issues regarding the \texttt{root-config} command, which calls the ROOT library.

If the root is installed with the RPM from the linux distribution, in \texttt{Makefile}: \\
\longindent {\tt shell \$(ROOTSYS)/bin/root-config --libs}

will not function. Thus, it needs to be changed to \\
\longindent {\tt shell  /root-config --libs}

then problem is fixed. User may or may not be required install extra root packages from the distribution. The self compiled version of ROOT should not have this issue 






\section{Configurations \& Trouble shoot}

Note: this section is most important .



\subsection{Setup Multiple Experiments}
There are many ways to setup experiments, I would introduce our way of setting up.

The main experiment directory is in:
\$HOME/experiment

And there are many sub experiments under this dir.
There is main exptab file under
\$HOME/experiment/exptab
which collects the information for all sub experiments, and each sub experiments have their own exptab which only contain sub experiments info. 
 

\subsection{Exptab File}

Exptab file setting is extremely important.

It contains the experiments information of Experiment name, Experiment Dir and User name. 	  

Example:\\ 
\longindent {\tt gaintest ~~/home/midasdaq2/experiment/gaintest ~~ midasdaq2 \\
	         simpletest /home/midasdaq2/experiment/simple\_test midasdaq2 \\
             muon-test ~/home/midasdaq2/experiment/muon08-test midasdaq2}\\

The main exptab file (under \texttt{\$HOME/experiment/}) should have information for all sub experiments, the exptab file in sub experiments should only contain its information.




\subsection{Sourcing steup.sh}

% File sourcing when \\
% \longindent \texttt{\$ ./start\_daq.sh}
 
When data acquisition system starts, the \texttt{start\_daq.sh} script will try to source the setup.sh, one MUST check if the \texttt{setup.sh} is set up correctly. 

Always set \texttt{\$MIDAS\_EXPTAB}: \\
\longindent \texttt{export MIDAS\_EXPTAB=./exptab}


Other setting are optional.



\subsection{Resolve the Root Privilege Issue}

With CAMAC crate, in order to access the controller, one need to use
\$dio ./frontend
command to bypass the driver. If Midas is not install on root account, dio owner must be changed to root.

If 64 bits do:\\
\longindent \texttt{
\$ su -c "chown root \$MIDASSYS/linux/bin/dio"\\
\$ su -c "chgrp root \$MIDASSYS/linux/bin/dio"}

and 32 bits do:\\
\longindent \texttt{
\$ su -c "chown root \$MIDASSYS/linux-m32/bin/dio"\\
\$ su -c "chgrp root \$MIDASSYS/linux-m32/bin/dio"}

Then set user or group ID on execution \\
\longindent \texttt{
\$ su -c "chmod a+s \$MIDASSYS/linux/bin/dio"}

or \\
\longindent \texttt{
\$ su -c "chmod a+s \$MIDASSYS/linux-m32/bin/dio"}

Warning: If this procedure is not done, dio will require root privilege to execute. Which implies no other user is able to run frontend and collect data!






\section{Get started with an example experiment}
We will go through a example experiment to test all MIDAS configuration. 

\begin{enumerate}
\item Copy {\tt \$MIDASSYS/examples/experiment/} to \\ {\tt \$HOME/experiment/testexperiment}
\item Edit both {\tt exptab} files
\item check {\tt testexperiment/setup.sh} (IMPORTANT), comment out \texttt{\$MIDASSYS} option
\item {\tt \$ make}
\item {\tt \$ ./startdaq}
\item {\tt \$ firefox localhost:8081}
\end{enumerate}




\section{MIDAS Analyser} 
For some reason, the MIDAS Analyser has default crazy pedestal values for all ADC channels, one must check the before running MIDAS, to verify:
{\tt 
\begin{lstlisting}
    $ cd PATH/To/Experiment
    $ odedit
    $ cd "Analyzer/Parameters/ADC calibration/"
    $ ls
\end{lstlisting}
}
Make sure that Pedestal values are set to 0, if not\\ 
\longindent \texttt{\$ set Pedestal[*] 0}


\subsection{Covert .mid file to .root File}  
When converting .mid to .root file, the analyser pedestal will be overwrote by the pedestal settings during the data taking. If the data pedestal is set to be correct, execute
{\tt
\begin{lstlisting}
    $ analyser -i runXXXX.mid -o runXXXX.root
\end{lstlisting}}

And data were taken with wrong pedestal, reset the pedestal values and use \texttt{-P} option to prevent the pedestal values being overwritten, an example is given as: 
{\tt
\begin{lstlisting}
    $ analyser -i runXXXX.mid -o runXXXX.root -P "/Analyzer/
      Parameters/ADC calibration" 
\end{lstlisting}}


\subsection{Covert .mid file to .asc file}  

To convert .mid file to .asc file, execute\\ 
\longindent \texttt{\$ analyser -i runXXXX.mid -o runXXXX.asc}



\end{document}
