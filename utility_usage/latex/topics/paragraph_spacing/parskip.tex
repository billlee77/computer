\documentclass[letterpaper, 10pt] {article}

\usepackage{parskip}


\begin{document}


\setlength{\parskip}{0.5cm plus4mm minus3mm}

\section{Supper Conducting Electron Linac: CEBAF}
Thomas Jefferson National accelerator Facility (JLab), also known as Continuous Electron Beam Accelerator Facility (CEBAF), is the world's largest superconducting RF linear accelerator facility. Its reliable continuous electron beam has become the most important method to probe the quark-gluon nature of nuclear matter at high energy. The CEBAF consists of an electron injector, a pair of superconducting linear accelerators and several bending arcs. 



The electrons are generated in the injector, then are grouped into micro-bunches and released into the north Linac at an energy of 45MeV with 0.667ns separation. The north and south Linacs have identical design, each hosts 20 cryomodules with accelerating gradient of 5MeV/m. The Linacs are connected by binding arcs at both end, the entire setup looks like a racetrack ring. Each electron bunch has to complete 5 circles in the racetrack ring before reaching the experimental area, the energy gain is 1GeV per circle thus the total energy gain for an electron is 6GeV. JLab is currently undertaking the 12GeV upgrade project to double its accelerating energy. 


\end{document}

