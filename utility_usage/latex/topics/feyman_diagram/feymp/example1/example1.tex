\documentclass{article}
\usepackage{feynmp}
\begin{document}
\unitlength = 1mm
% determine the unit for the size of diagram.
... here comes an example

\begin{fmffile}{simple}
\begin{fmfgraph}(40,25)
% Note that the size is given in normal parentheses
% instead of curly brackets.
% Define external vertices from bottom to top
\fmfleft{i1,i2}
\fmfright{o1,o2}
\fmf{fermion}{i1,v1,o1}
\fmf{fermion}{i2,v2,o2}
\fmf{photon}{v1,v2}
\end{fmfgraph}
\end{fmffile}



\begin{fmffile}{simplelabels}
\begin{fmfgraph*}(40,25)
\fmfleft{i1,i2}
\fmfright{o1,o2}
\fmflabel{$e^-$}{i1}
\fmflabel{$e^+$}{i2}
\fmflabel{$e^+,\mu^+$}{o1}
\fmflabel{$e^-,\mu^-$}{o2}
\fmflabel{$i\sqrt{\alpha}$}{v1}
\fmflabel{$i\sqrt{\alpha}$}{v2}
\fmf{fermion}{i1,v1,i2}
\fmf{fermion}{o1,v2,o2}
\fmf{photon,label=$\gamma,,Z^0$}{v1,v2}
\end{fmfgraph*}
\end{fmffile}
Note that you need two commas inside \fmf command






\end{document}
