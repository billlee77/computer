\documentclass[12pt]{article}
\usepackage{latexsym,amssymb,amsmath,times, nopageno}

\topmargin 1in
\oddsidemargin 5mm
\evensidemargin 5mm
\textwidth 165mm
\textheight 200mm

\begin{document}

\begin{center}
{\bf ~\\\huge
Cherenkov Mirror Reflectivity Measurements for Hall C at Jefferson  Lab
}
\end{center}


\begin{center}
Wenliang Li \\[2mm]

University of Regina \\[3mm]

\end{center}

\normalsize
\noindent
Jefferson Lab (JLab) has undertaken the 12 GeV Upgrade to double the accelerating energy of its electron beam. This attracts many interesting proposals to probe the quark-gluon nature of nuclear matter at higher energy, therefore a new set of experimental equipment is required. Hall C of JLab has planned to construct a new Super High Momentum Spectrometer (SHMS) to replace the existing Short Orbit Spectrometer (SOS). The University of Regina is assigned to construct the Heavy Gas Cherenkov (HGC) Detector as part of the SHMS focal plane detectors. The HGC consists of four aluminized mirrors which are required to reflect 70-90\% of UV photons between 200-400 nm wavelength. In this talk, we present latest SHMS HGC mirror reflectivity results obtained with the reflectivity measurement setup at JLab. This setup uses the lock-in amplification technique, which uses a photo-diode instead of a PMT, and is capable of measuring large size optics up to 60 cm X 55 cm.





\end{document}
