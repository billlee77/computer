\documentclass[12pt, letterpaper]{article}

\usepackage{times}
\usepackage{amsmath}
\usepackage{nopageno}
\usepackage{setspace}
\usepackage{fullpage}

\title{Statement of Research Activity}
\author{Wenliang Li}

\begin{document}
\maketitle
\onehalfspacing

\paragraph{Research Overview}
The top priority of all particle physicists is to seek the fundamental building
blocks of matter, and further understand the interaction between
them. Right now, the smallest building blocks of our world are known as quarks
and leptons; however, the answer to the question:\\
\emph{How do quarks interact inside of proton and neutrons?} \\ 
remains a mystery.  The standard method to research this subject is to use
electron scattering, i.e. to accelerate electrons at high momenta and collide
with a hydrogen target then look at the products after the interaction.

The Thomas Jefferson National Accelerating Facility (JLab), located in Newport
News, Virginia, is the world leading electron accelerating facility.  It has
been used by scientists around the world for research in subatomic and medical
physics. In 2008, JLab was approved by the U.S. Department of Energy for a
U.S.\$310 million upgrade to increase the maximum electron beam energy from 6 GeV to
12 GeV and build new experimental facilities.  This upgrade is in progress
with a projected completion date of 2015.

\paragraph{Personal Research Role}
The JLab 12 GeV upgrade calls for new set of experimental equipment to detect
produced particles at higher momenta.  In Experimental Hall C of JLab, the
Super High Momentum Spectrometer (SHMS) will be installed as part of the
upgrade.  The essential role of the SHMS is to allow scientists to distinguish
different particle types and precisely measure their momenta and angles of
emission. A part of the SHMS instrumentation known as the Heavy Gas Cerenkov
Detector used for particle identification is being constructed at the
University of Regina with funds provided by the Government of Canada.  My
supervisor at the University of Regina, Dr. Garth Huber, has over 20 years of
research experience at JLab, and his work on the \emph{Charged Pion Form
Factor} is one of the signature results to come from the Hall C Group at
JLab. My work is to assist Dr. Garth Huber with the construction of the Heavy
Gas Cerenkov Detector, and to perform experiments to test and the calibrate
detectors at the early running stage.

\end{document}



