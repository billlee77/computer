\documentclass[12pt, letterpaper]{article}

\usepackage{times}
\usepackage{amsmath}
\usepackage{nopageno}
\usepackage{setspace}
\usepackage{fullpage}

\title{Statement of Research Activity}
\author{Wenliang Li}

\begin{document}
\maketitle
\onehalfspacing

\paragraph{Research Overview}
The top priority of all particle physicists is to seek the Fundamental Building Blocks of the matter, and further understand the interaction between them. Right now, the smallest building blocks of our world are known as quarks and leptons, however, the answer to question:\\ 
\emph{How do quarks interact inside of proton and neutrons?} \\
remain a mystery. The standard method to research this subject is to use method of electron scattering, i.e. to accelerate electron at high momenta and collide with Hydrogen target, then look at the product after the interaction.

Thomas Jefferson National Accelerating Facility located in Newport, Virginia is the world leading electron accelerating facility, it has been used by scientists around the world for research in subatomic, and medical physics. In 2008, Jlab received \$310 for a new upgrade plan to increase the maximum electron beam from 6 GeV to 12 GeV.  


\paragraph{Personal Research Role}
The 12 GeV upgrade at the Thomas Jefferson National Accelerator Facility (JLab) in Virginia calls for new set of experimental equipment to detect produced particles at higher momenta.  In Experimental Hall C of JLab, the Super High Momentum Spectrometer (SHMS) will be installed as part of the upgrade.  The essential role of SHMS is to allow scientists to distinguish different particle types and measure their momenta and angles of emission. A part of the SHMS instrumentation known as the Heavy Gas Cerenkov Detector used for particle identification is being constructed at the University of Regina. 
Dr. Garth Huber has over 20 years of research experience associates with Jlab, his work on \emph{Charged Pion Form Factor} was one of the signature result came from Hall C Group at Jlab. My work is to assist Dr. Garth Huber to construct Heavy Gas Cerenkov Detector, and perform experiment to test and calibrate detectors at the early running stage.

\end{document}



