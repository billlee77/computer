\documentclass[12pt, letterpaper]{article}

\usepackage{times}
\usepackage{amsmath}
\usepackage{nopageno}
\usepackage{setspace}
\usepackage{fullpage}

\title{Statement of Research Activity}
\author{Wenliang Li}

\begin{document}
\maketitle
\onehalfspacing

\section{Past Research}


\subsection*{Three month internship at Karlsruhe Institute of Technology (KIT)}

\paragraph{Research Description} During three month internship at Kalsruhe Institute of Technology (KIT) in Germany, I participated a Pierre Auger Experiment related research project was titled \emph{Star Tracking with the High Elevation Auger Telescopes}. A simulation was constructed to predict the star trajectory in the field of view of the florescences telescopes. Then this result was compared with detected star trajectory by florescences telescopes. The simulation-data difference can be used for many different applications, and one of them was to identify mis-cabled PMTs. In the final period of the internship, a mis-cabled PMT checking algorithm was developed and it was proven to be successful. 

\paragraph{Computer Package Used} Offline, C++, ROOT, SVN.


\subsection*{Final year project at University of Kent at Canterbury}
%\begin{itemize}
\paragraph{Research Description} The research I carried out at University of Kent as final year project was tittled \emph{Investigation of Far Ultraviolet (FUV) radiation induced shock-wave velocity in H2 region}. My task was to use the existing Smoothed Particle Hydrodynamics (SPH) simulation to investigate the affect of the FUV radiation on the molecular clouds. Clouds of different radius and mass were studied. 

\paragraph{Computer Package Used} IDL, C++, ROOT, FORTRAN 90, SVN, FFMPEG.
%\end{itemize}




\section*{Current Research}


% \paragraph{Research Overview}
% The top priority of all particle physicists is to seek the fundamental building blocks of matter, and further understand the interaction between them. Right now, the smallest building blocks of our world are known as quarks and leptons; however, the answer to the question:\\
% \emph{How do quarks interact inside of proton and neutrons?} \\ 
% remains a mystery.  The standard method to research this subject is to use electron scattering, i.e. to accelerate electrons at high momenta and collide with a hydrogen target then look at the products after the interaction.
% 
% The Thomas Jefferson National Accelerating Facility (JLab), located in Newport News, Virginia, is the world leading electron accelerating facility.  It has been used by scientists around the world for research in subatomic and medical physics. In 2008, JLab was approved by the U.S. Department of Energy for a U.S.\$310 million upgrade to increase the maximum electron beam energy from 6 GeV to 12 GeV and build new experimental facilities.  This upgrade is in progress with a projected completion date of 2015.



%The JLab 12 GeV upgrade calls for new set of experimental equipment to detect produced particles at higher momenta. In Experimental Hall C of JLab, the Super High Momentum Spectrometer (SHMS) will be installed as part of the upgrade.  The essential role of the SHMS is to allow scientists to distinguish different particle types and precisely measure their momenta and angles of emission. A part of the SHMS instrumentation known as the Heavy Gas Cerenkov Detector used for particle identification is being constructed at the University of Regina with funds provided by the Government of Canada. My supervisor at the University of Regina, Dr. Garth Huber, has over 20 years of research experience at JLab, and his work on the \emph{Charged Pion Form Factor} is one of the signature results to come from the Hall C Group at JLab. 

\paragraph{Research Overview}
In the master program, I assisted Dr. Garth Huber at University of Regina with the construction of the Heavy Gas Cerenkov Detector, which is part of the 12 GeV upgrade installation at Jefferson Lab, VA. I was involved in detector design, R\&D , Monte-Carlo simulation, and hardware quality control. I also participated in the construction of the permanent facility for reflectivity measurement at Jefferson Lab. The setup was successful. 


% and to perform experiments to test and the calibrate detectors at the early running stage.
 



\section*{Future Research Goal}
\paragraph{Motivation} The master experience at University of Regina in particle physics was very enjoyable and productive, it broadened my knowledge and skills in physics and life. I am willing to continue on the experimental particle physics study and research as PhD student, and believe this experience will bring me one step closer to being a real physicist. Also, it will give to great joy to see the installation and commission the HGC detector.  

%I am determined to pursue the final target as a student and succeed.

%The interests towards this subject was developed gradually during the undergraduate study, due to the demand of the understanding the behaviour of elementary particles.  I have decided to take unsolved question a step further to the research level, to boost my knowledges as well as discovering new science. 

%\paragraph{PhD Research} For the PhD research, I would desire any topic in studying the elementary particle physics, such as quark model, QCD and QED, etc.


\end{document}



